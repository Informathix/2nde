\documentclass[10pt,fleqn]{article} % Default font size and left-justified equations
\usepackage[%
    pdftitle={SNT : Image},
    pdfauthor={Julien Chu - Cédric Lusseau - Fadi Tamin}]{hyperref}
\usepackage{amsmath,amssymb}  
\usepackage[table]{xcolor}  
\usepackage{tcolorbox}
\usepackage{wrapfig,graphicx}
\usepackage{wasysym}
\tcbuselibrary{skins,listingsutf8,xparse}


\input{style/new_style}
\input{style/macros_SII}

%\fichetrue
\fichefalse

%\proftrue
\proffalse

%\tdtrue
\tdfalse

\courstrue
%\coursfalse

% -------------------------------------
% Déclaration des titres
% -------------------------------------

\def\discipline{SNT}
\def\xxtete{SNT}

\def\classe{Seconde}
\def\xxnumpartie{}
\def\xxpartie{Image}

\def\xxnumchapitre{Chapitre 2}
\def\xxchapitre{\hspace{.12cm} Image}

\def\xxposongletx{2}
\def\xxposonglettext{1.45}
\def\xxposonglety{22}%19

\def\xxonglet{Image}

\def\xxactivite{Cours}
\def\xxauteur{\textsl{Julien Chu - Cédric Lusseau - Fadi Tamin}}

\def\xxcompetences{%
\textsl{%
\textbf{Savoirs et compétences :}
\begin{itemize}[label=\ding{112},font=\color{ocre}] 
\item Expliciter la Prise de vue et le fonctionnement des Capteurs d'un appareil photo numérique
\item Retrouver les métadonnées d'une photographie
\item Identifier les étapes de la construction de l'image finale
\item Traiter par programme une image pour la transformer
\end{itemize}
}}

\def\xxfigures{
}%figues de la page de garde

\def\xxpied{Ch 2 : Image -- \xxactivite%
}



\usepackage{framed}
\definecolor{violetf}{RGB}{112,48,160}
\definecolor{violetc}{RGB}{230,224,236}
\newenvironment{oldpy}[1][\hsize]%
{%
    \def\FrameCommand%
    {%
%\rotatebox{90}{\textit{\textsf{Python}}}
\rotatebox{90}{\includegraphics[height=.6cm]{png/logo_python}}
        {\color{violetf}\vrule width 3pt}%
        \hspace{0pt}%must no space.
        \fboxsep=\FrameSep\colorbox{violetc}%
    }%
    \MakeFramed{\hsize #1 \advance\hsize-\width\FrameRestore}%
}%
{\endMakeFramed}%

%---------------------------------------------------------------------------
\begin{document}
\chapterimage{png/Fond_ALG}
\input{style/new_pagegarde}
%---------------------------------------------------------------------------

\section{Introduction -Débat}

\subsection{un peu d'histoire}
\begin{multicols}{2}
\begin{flushleft}

	\includegraphics[width=0.5\textwidth]{Images/histoirephoto-01.jpeg}
	\\
	\textit{Début de la photo}

	\includegraphics[width=0.6\textwidth]{../Frises/frise-photo-SNT.png}
\end{flushleft}


\end{multicols}

Le mot « photographie » signifie littéralement « dessin avec de la lumière ». Le mot aurait été inventé par le scientifique britannique Sir John Herschel en 1839 à partir des mots grecs phôs (génitif: phôtos) signifiant «lumière» et graphê signifiant «dessin ou écriture». La technologie qui a conduit à l’invention de la photographie combine deux sciences distinctes : l’optique, avec la convergence des rayons lumineux pour former une image à l’intérieur d’une caméra, et la chimie, pour permettre à cette image d’être capturée et enregistrée en permanence sur un support photosensible (sensible à la lumière).



\subsection{Les dérives de la photographies}
Il est possible d'interpréter différemment les données produites par les capteurs d'un appareil numérique puis de traiter l'image en post production. Il peut en découler des images truquées et potentiellement dangereuses. Rien de plus facile, à partir d'une image réelle, de créer une fausse information (fake news) ou un canular (HOAX). Les buts poursuivis peuvent être multiples de la blague potache ou la tentative de faire du "buzz" à la propagande politique ou idéologique en passant par le message publicitaire ou une tentative d'extorsion d'argent.
Mais, même sans retoucher une image, il est possible de nuire, parfois fortement, à des personnes en publiant des photos à leur insu, principalement sur les réseaux sociaux.
Quelques exemples de dérive.
\begin{itemize}
\item 1942 : pas de numérique mais déjà un peu de propagande. Mussolini, en quête de grandeur, fait retirer l’écuyer…
	\includegraphics[width=0.5\textwidth]{Images/derive1.png}
\item Un peu d’humour ?
\\
	\includegraphics[width=0.5\textwidth]{Images/derive2.png}
\item Un mensonge?
\begin{multicols}{2}

En 2013, cette étudiante a passé cinq semaines en Asie du Sud-Est. C'est du moins ce que son entourage a cru. Grâce à d'habiles falsifications de photos et autres stratagèmes, elle a réussi à convaincre ses proches qu'elle parcourait la Thaïlande, le Cambodge et le Laos, alors qu'elle était en fait chez elle, à Amsterdam. 

\begin{flushleft}
\includegraphics[width=0.3\textwidth]{images/derive3.png}
\end{flushleft}
\end{multicols}
\item Cet homme n’existe pas !

\begin{multicols}{2}

Les IA sont capables de créer un visage à partir d'autres. Il est même possible de remplacer tout ou partie du visage d'une personne par celui d'une autre dans une vidéo. De quoi créer des faux discours à la pelle.

\begin{flushleft}
\includegraphics[width=0.3\textwidth]{images/derive4.png}
\end{flushleft}
\end{multicols}

\item Photomontage ? Les bons réflexes.

\begin{multicols}{2}

\includegraphics[width=0.3\textwidth]{images/QR1.png}
	\textit{FAKE NEWS : les gestes qui sauvent - Comment vérifier une photo ?}
	\href{https://youtu.be/lMWPDuiZ4L8}{https://youtu.be/lMWPDuiZ4L8}
	
\includegraphics[width=0.3\textwidth]{images/QR2.png}
	\textit{Comment être sûr qu’une image trouvée sur Internet est authentique ?}
	\href{https://youtu.be/rN_Vh9wZEyI}{https://youtu.be/rN$\_$Vh9wZEyI}

\end{multicols}
\item Le vrai Obama prononce un faux discours : un trucage criant de vérité
	\\Des chercheurs ont créé, à partir de bandes audio, une vidéo dans laquelle Barak Obama donne un discours qu'il n'a jamais prononcé.
	
	\includegraphics[width=0.5\textwidth]{Images/derive5.png}
\includegraphics[width=0.3\textwidth]{images/QR3.png}
	\href{https://youtu.be/UCwbJxW-ZRg}{https://youtu.be/UCwbJxW-ZRg}	

\item Lutter contre ces dérives	

Il existe des applications capables de faire des recherches, comme un moteur de recherche classique, mais avec des photographies. C'est le cas de tinEye par exemple.

	\includegraphics[width=0.5\textwidth]{Images/derive6.png}

\item 	Forensically : pour détecter des retouches d'images. A tester !
\\
	\includegraphics[width=0.5\textwidth]{Images/derive7.png}
		\href{https://29a.ch/photo-forensics/#forensic-magnifier}{https://29a.ch/photo-forensics}	
	
\item le prix de la gratuité des photos numériques

La facilité de prise de vue et de partage des photographies ne doit pas masquer le coût  réel de la gratuité apparente. Le nombre de photographie augmente de manière importante 
Le stockage à long terme d'une photographie a aussi un coût énergétique et environnemental non négligeable. 

Cette course au stockage et au traitement des données semble sans fin dans une consommation des services offerts par le réseau mondial toujours grandissante. La pratique du selfie par exemple, qui peut sembler anodine, est en réalité un gouffre énergétique et écologique: chaque photo prise par un utilisateur de smartphone pour son mur facebook est envoyée à travers des dizaines de milliers de kilomètres de fibres et de câbles, transitant par des équipements réseaux jusqu'aux data centers surdimmensionnés de facebook et consomme à elle seule autant que 3 ou 4 ampoules de 20 Watts allumées pendant une heure! Puis vient ensuite l'alimentation des serveurs de stockage qui conservent le 'précieux cliché" et leur climatisation. Un selfie Facebook est à lui seul une petite entreprise de consommation énergétique. Il y a deux milliards de compte Facebook et une grande partie des utilisateurs postent plusieurs photos par jour. Combien de data centers va-t-il falloir construire et alimenter dans les années qui viennent pour satisfaire cette seul demande?

\end{itemize}

\subsection{Une situation}

\RIGHTarrow ~~Il vous arrive parfois (souvent?) de publier vos photos sur les réseaux sociaux .\\
Mais ses photos ne se trouvent pas toujours aux endroits que vous pensez!\\
Voici ce qu'écrit un utilisateur sur le blog d'un site dédié à la photo(le prénom et la date ont été changés):

\begin{tcolorbox}[colframe =
	orange, colback = orange!50,width=17cm,
	boxrule = 2pt, arc = 6pt,
	title = {}, coltitle
	= black]
	\vspace{0.3cm}
	
	Paul ,	le 25 mars 2018 à 09 h 27 min
	
	\medskip
	
	
	Bonjour,
	J’ai participé à un concours photos et ma photo a été sélectionnée et donc visible sur internet.\\
	Je me rends compte récemment qu’un site de grande audience a utilisé cette photo.\\
	Mon nom est mentionnée cependant à aucun moment, je n’ai donné mon accord. Et j’imagine que cette photo peut-être utilisée par d’autres.\\
	Suis-je en mesure de faire quelque chose?\\
	Merci à vous.\\
	Paul
\end{tcolorbox}

\begin{enumerate}
	\item 
	
	
	Avez-vous vécu ce genre d'expériences?\\
	Comment avez-vous découvert qu'une de vos photos , vidéos , se sont retrouvés sur la toile ? Sur quel genre de sites?\\
	Peut-on vérifier soi-même où se trouvent nos photos , vidéos?
	
	% On peut proposer aux élèves de chercher eux-mêmes
	% une façon (en utilisant Google par exemple)
	
	
	
	\item
	
	Sur votre smartphone , sélectionner une photo .\\
	Votre appareil a enregistré quelques données EXIF.\\
	Pouvez-vous les récupérez?Pouvez-vous les modifier?\\
	A-t-on la même chose quel que le modèle du smartphone?
\end{enumerate}


\section{De la prise de vue à l'image numérique}

\subsection{Les grandes étapes et les principaux outils}

\begin{itemize}
	\item La prise de vue :Capteur de l'appareil photo numérique,réglages de certains composants physiques de façon manuelle ou automatique  (ce qui nécessite des algorithmes).
	
	Des algorithmes permettent de traiter toutes les lumières, d’effectuer une retouche facile,
	avec une qualité maintenant bien supérieure à l’argentique. Avec l’arrivée du téléphone
	mobile, des algorithmes de fusion d’images permettent de concilier une excellente qualité
	avec un capteur et un objectif minuscules.
	De nombreux algorithmes sophistiqués sont utilisés dans les appareils de photographie
	numérique :
    Lors de la prise de vue : calcul de l’exposition, mise au point, stabilisation par le
	capteur et/ou l’objectif, le tout en automatique ou manuel assisté, \textbf{focus-peaking}
	(scintillement des contours nets), prise en rafales rapides d’images multiples avant et
	après appui sur le déclencheur.
    \item Lors du développement de l’image issue du capteur en une image pixellisée : gestion
	de la lumière et du contraste, balance des blancs, netteté, débouchage des ombres,
	correction automatique des distorsions ou des aberrations optiques.
    \item  Après le développement : compression du fichier (TIFF sans perte, JPEG avec
	perte).
    \item En utilisant la fusion d’images : réduction du bruit et amélioration de la netteté,
	panoramas,\textbf{ HDR} (High Dynamic Range), super-résolution par micro-décalages du
	capteur, focus stacking pour étendre la netteté avec plusieurs mises au point
    \item Certains appareils peuvent augmenter leurs fonctionnalités par téléchargement de
	nouveaux logiciels.
	\item Exploitation de l'image :Visualisation ,publication sur des sites , stockage dans le cloud puis retouche éventuelle de l'image grâce à des logiciels (encore des algorithmes).
\end{itemize}

\subsection{activité sur la prise de vue}
Pour appréhender le vocabulaire de la prise de vue nous utiliserons l'outil pédagogique développé par l'UTC. Cliquez sur le lien puis sur "Jouer". Compléter ensuite les différents réglages.

\includegraphics[width=0.7\textwidth]{Images/declenchemoi.png}
\\
http://www.utc.fr/rendezvouscreation/francais/connaissances/outilspedagogiques/declenchezmoi/files/index.html
%	\href{http://www.utc.fr/rendezvouscreation/francais/connaissances/outilspedagogiques/declenchezmoi/files/index.html}	

\begin{defi}
\textbf{Sensibilité}
Plus l’appareil photo est sensible, moins il aura besoin de recevoir de lumière. Par beau temps, une grande quantité de lumière pourra être capturée en peu de temps, une faible sensibilité sera généralement suffisante. Par ailleurs, plus la sensibilité est élevée, plus le gain ou le bruit sera visible. Plus la sensibilité est élevée, plus la valeur qui la désigne sera importante. Par exemple, 100 ISO est une faible sensibilité alors que 1600 ISO est une sensibilité élevée.
\end{defi}




\begin{itemize}
\item Prenez une photo en utilisant la plus grande sensibilité disponible.
\\ \includegraphics[width=0.7\textwidth]{Images/resultat.png}
\item Prenez une photo présentant le moins de bruit possible.
\\ \includegraphics[width=0.7\textwidth]{Images/resultat.png}
\end{itemize}


\begin{defi}
\textbf{Balance des blancs }
Régler la balance des blancs correspond à indiquer à un appareil photo numérique la couleur de l’éclairage. En indiquant effectivement la couleur de l’éclairage, le rendu des couleurs sera fidèle. Si ce n’est pas le cas, le rendu sera soit plus chaud (jauni/ rougi) soit plus froid (bleuté).
La balance des blancs peut être automatique, prédéfinie (on choisit parmi une liste comme : lumière du jour, ampoule au tungstène ; tube fluorescent, etc…) ou manuelle (via une valeur en Kelvin ou une photo de référence). La photo de référence se fait généralement sur une charte grise, l’appareil en déduira la couleur de l’éclairage.
\end{defi}


\begin{itemize}
\item Prenez une photo présentant le rendu le plus froid possible.
\\ \includegraphics[width=0.7\textwidth]{Images/resultat.png}
\item Prenez une photo présentant un rendu fidèle des couleurs.
\\ \includegraphics[width=0.7\textwidth]{Images/resultat.png}
\end{itemize}

\begin{defi}
\textbf{Mesure de la lumière et correction de l’exposition}

Quand il n’est pas en mode manuel, l’appareil photo mesure la lumière de la scène pour fournir une image à la luminosité moyenne (pas trop claire, pas trop sombre). Si l’on n’est pas satisfait du résultat, on peut demander à l’appareil une correction positive (plus clair) ou une correction négative (plus sombre)
La mesure de la lumière peut être effectuée de plusieurs manières, principalement : mesure multizone(l’appareil tient compte de différentes zones couvrant toute l’image) pondération centrale (focalisation du centre) et spot (seulement quelques pourcents de l’image au centre, très utile en contre jour par exemple). La correction d’exposition correspond à un nombre positif (par exemple +1, très utile pour corriger une sous exposition dans un paysage blanc) ou négatif (par exemple -2)

\end{defi}


\begin{itemize}
\item Obtenez une photo trop claire sans correction d’exposition ni mode manuel.
\\ \includegraphics[width=0.7\textwidth]{Images/resultat.png}
\item Obtenez une photo correctement exposée avec la mesure spot.
\\ \includegraphics[width=0.7\textwidth]{Images/resultat.png}
\end{itemize}


\begin{defi}
\textbf{Mode S et vitesse d’obsturation}


La quantité de lumière reçue lors d’une prise de vue est gérée par deux paramètres : la vitesse d’obturation (temps de pose) et l’ouverture du diaphragme (les lamelles qui viennent obsturer partiellement l’objectif). En mode M (manuel), le photographe indique lui-même ces deux paramètres. En mode P (programme), l’appareil les fournit automatiquement (en fonction de la mesure de la lumière). En mode S (priorité à la vitesse), le photographe indique sa vitesse et l’appareil fournit une ouverture (toujours en fonction de la mesure de la lumière).
La vitesse d’obturation s’exprime soit en fraction de seconde (par exemple  pour 1/50e de seconde) soit en seconde (par exemple 2’’ pour deux secondes). Une vitesse élevée (par exemple 1/2000 e de seconde) figera une action (très utile pour le sport). A l’opposée, une vitesse lente (par exemple : huit secondes) enregistrera un mouvement (par exemple : des feux de voiture qui passent dans la nuit).


\end{defi}


\begin{itemize}
\item Avec le mode S, figez le plus possible le mouvement du ventilateur.
\\ \includegraphics[width=0.7\textwidth]{Images/resultat.png}
\item Avec le mode S, obtenez le plus gros flou possible pour le mouvement du ventilateur.
\\ \includegraphics[width=0.7\textwidth]{Images/resultat.png}
\end{itemize}



\begin{defi}
\textbf{Mode A ouverture et profondeur de champ}

En mode A (priorité à l’ouverture), le photographe indique son ouverture et l’appareil fournit alors une vitesse (en fonction de la mesure de la lumière). Plus l’ouverture du diaphragme est grande, plus l’objectif laisse entrer la lumière. Par ailleurs, l’ouverture est l’un des paramètres qui permet de gérer la profondeur de champ (la zone de netteté dans l’image). Plus l’ouverture est grande, plus la profondeur de champ est faible. (très utile pour avoir un flou en arrière-plan).
L’ouverture se note avec un « f/ » suivi d’un nombre. Plus ce nombre est grand, plus l’ouverture est faible. Une grande ouverture (par exemple :f/2) impliquera une faible profondeur de champ (le plan de mise au point est net, l’arrière plan est flou). A l’opposé, une petit ouverture (par exemple : f/16) impliquera une grande profondeur de champ (plusieurs plans sont nets

\end{defi}


\begin{itemize}
\item Avec le mode A, obtenez la plus faible profondeur de champ possible.
\\ \includegraphics[width=0.7\textwidth]{Images/resultat.png}
\item Avec le mode A, obtenez la plus grande profondeur de champ possible.
\\ \includegraphics[width=0.7\textwidth]{Images/resultat.png}
\end{itemize}




\subsection{Les grandes étapes et les principaux outils}


\begin{tcolorbox}[width=16cm,enhanced,colback=lightgray!5!white,colframe=blue,
	attach boxed title to top left=
	{yshift=-\tcboxedtitleheight/2},
	boxed title style={size=small,colback=black}]

\vspace{0.3cm}
\begin{itemize}
	\item La prise de vue :Capteur de l'appareil photo numérique,réglages de certains composants physiques de façon manuelle ou automatique  (ce qui nécessite des algorithmes).
	
	Des algorithmes permettent de traiter toutes les lumières, d’effectuer une retouche facile,
	avec une qualité maintenant bien supérieure à l’argentique. Avec l’arrivée du téléphone
	mobile, des algorithmes de fusion d’images permettent de concilier une excellente qualité
	avec un capteur et un objectif minuscules.
	De nombreux algorithmes sophistiqués sont utilisés dans les appareils de photographie
	numérique :
    Lors de la prise de vue : calcul de l’exposition, mise au point, stabilisation par le
	capteur et/ou l’objectif, le tout en automatique ou manuel assisté, \textbf{focus-peaking}
	(scintillement des contours nets), prise en rafales rapides d’images multiples avant et
	après appui sur le déclencheur.
    \item Lors du développement de l’image issue du capteur en une image pixellisée : gestion
	de la lumière et du contraste, balance des blancs, netteté, débouchage des ombres,
	correction automatique des distorsions ou des aberrations optiques.
    \item  Après le développement : compression du fichier (TIFF sans perte, JPEG avec
	perte).
    \item En utilisant la fusion d’images : réduction du bruit et amélioration de la netteté,
	panoramas,\textbf{ HDR} (High Dynamic Range), super-résolution par micro-décalages du
	capteur, focus stacking pour étendre la netteté avec plusieurs mises au point
    \item Certains appareils peuvent augmenter leurs fonctionnalités par téléchargement de
	nouveaux logiciels.
	
	\item Exploitation de l'image :Visualisation ,publication sur des sites , stockage dans le cloud puis retouche éventuelle de l'image grâce à des logiciels (encore des algorithmes).
\end{itemize}

\vspace{0.3cm}
\end{tcolorbox}

\subsection{Rôle et influence des capteurs}


\begin{multicols}{2}

\begin{center}
	
	\includegraphics[width=0.5\textwidth]{Images/photosite.png}
	\textit{Photosite}
	
\end{center}

Visualisez  la vidéo suivante sur youtube :\\

\begin{center}
	\includegraphics[width=10cm,height=6cm]{Images/video-capture.png}
	\centering{\url{https://www.youtube.com/watch?v=eY4s1sVsiAM}}
\end{center}

\end{multicols}


\section{Les métadonnées}

\subsection{Le format EXIF}

\begin{center}
	\includegraphics[width=10cm,height=6cm]{Images/nenuphar04}
	\centering{ce n'est pas qu'une photo...}
\end{center}\begin{tcolorbox}[width=17cm,enhanced,colback=lightgray!5!white,colframe=red,title=Pour démarrer,
	attach boxed title to top left=
	{yshift=-\tcboxedtitleheight/2},
	boxed title style={size=small,colback=black}]
	~~\\
	\RIGHTarrow ~~Le format \textbf{EXIF}(exchangeable Image File) a été développé par la \textbf{JEIDA} (Japan Electronic Industry evelopment Association).\\Tous les fabricants l'utilisent mais ce n'est pas un standard .\\
	C'est un ensemble de métadonnées ajoutées aux images produites~\cite{Tache} par les appareils de photo numérique (soit par les appareils eux-mêmes soit par le photographe).\\
	\vspace{0.4cm}
	Exemples de métadonnées Exif:
	\begin{itemize}
		\item Date et heure de la prise de vue.
		\item Réglages de l'appareil au moment de la prise de vue.
		\item Dimensions et résolution de l'image.
		\item Coordonnées GPS du lieu de la prise de vue.
		\item Modèle et fabricant de l'appareil.
	\end{itemize}
	
\end{tcolorbox}


\subsection{Comment lire les métadonnées ?}


\RIGHTarrow ~~La lecture des données Exif se fait à l'aide de logiciels de visualisation ou de traitement d'images.
\medskip

Des modules en python permettent également une lecture détaillée des métadonnées (voir activités ci-dessous).

\medskip

Des outils en ligne existent également.
Une manière simple est d'effectuer un clic droit de la souris sur le fichier de l'image et d'ouvrir les \textbf{propriétés} de l'image.\\
Mais des logiciels plus spécialisés permettent d'avoir l'accès à plus d'informations:\\ (\textbf{GIMP,~Photoshop,~Photofiltre etc})
\\

\RIGHTarrow ~~Lecture en ligne:\\
Le site ~~ \underline{{\url{http://exif.regex.info/exif.cgi}}} ~~permet de lire les métadonnées d'une image.\\
Importer la photo "image15.jpeg" et visualisez les données EXIF.\\

Relever notamment la date de la prise de vue ...

\begin{tcolorbox}[enhanced,title=\textit{Utilisation des métadonnées à des fins d'investigation},
	attach boxed title to top center=
	{yshift=-\tcboxedtitleheight/2},
	boxed title style={colback=black}]
	~~\\
	
	
	\begin{large}~~[..] il est primordial de connaître le modèle d’appareil ayant permis la prise de vue, ainsi que les réglages éventuellement opérés par l’auteur de la photo, de façon à pouvoir retrouver les paramètres techniques indispensables à l’analyse ~\cite{Mort}. Ces informations sont en général fournies par les données auxiliaires – ou métadonnées – associées au format standard des fichiers images générés par un appareil photo numérique.\\
		
		
		Ces métadonnées EXIF sont supportées par tous les formats de fichiers images, à l’exception de JPEG2000 et PNG. Elles comprennent de nombreuses balises, définissant aussi bien les données de dates, de caractéristiques techniques de l’appareil photo utilisé, de géolocalisation, de droits d’auteur ou encore de programmes tiers utilisés pour modifier le fichier (Photoshop…).Comme vu dans~\cite{Voyage}
		
		
		Le nombre de données EXIF effectivement fournies, ainsi que leur disposition, varie considérablement d’un appareil photographique à l’autre, voire même, pour un appareil donné, selon le micrologiciel (« firmware ») utilisé.\\
		
		
		Malheureusement, l’étude seule de ces métadonnées ne suffit pas à démontrer la présence d’une éventuelle supercherie. En effet, elles sont facilement modifiables et peuvent également être intégralement remplacées grâce à l’utilisation de logiciels spécialisés, tels qu’EXIFer, EXIFtool ou même un simple éditeur de texte en hexadécimal.\\
		
		
		Une bonne approche cependant - et un travail préliminaire utile - consiste à comparer les données EXIF du document photographique étudié avec celles extraites d’une autre photographie prise avec le même appareil, si possible comportant le même firmware. Dans un grand nombre de cas, cette simple vérification permet de mettre en évidence l’utilisation d’un programme de retouche.
		
		\medskip
		
		
		
		Sur un document qui n’est pas original, cela n’a aucune valeur intrinsèque, tant il existe diverses façons involontaires de modifier les métadonnées. Cependant, sur un document de première main et original (ou affirmé comme tel par le témoin), cela peut permettre l’invalidation de l’authenticité du document.
		
		\vspace{0.3cm}
		
		(\textit{extraits de \url{http://www.ipaco.fr/page7.html}})
	\end{large}
	
\end{tcolorbox}


\subsection{Activités en \textit{python}}


Le module \textbf{exifread } du langage \textit{python} permet de récupérer les métadonnées Exif (voir activité 1).\\
D'autres logiciels permettent également de modifier certaines métadonnées ou d'en rajouter (voir ci-dessous).

\vspace{0.4cm}
\textbf{Activité 1}


	Lire les métadonnées de l'image "photo11.jpg" située dans le dossier "fichiereleve":\\
	
	
	\medskip
	
	Pour ceci , écrire (on l'adaptera en fonction du nom du fichier) puis exécuter le programme suivant dans pyzo:
	
	
	
	
	
\begin{oldpy}
	\begin{python}
import exifread 		

f=open("photo12.jpg","rb") 		# Ouvrir le fichier image 
tags=exifread.process_file(f)	# On retourne les  tags EXIF
for tag in tags.keys():			# On affiche les donnees EXIF
    if tag in ('Image ImageWidth', 'Image ImageLength', 'Image DateTime','GPS GPSLatitude','GPS GPSLongitude'):
        print ( "Key: %s, value %s" % (tag, tags[tag]))		
	
\end{python}
\end{oldpy}

\begin{enumerate}
	\item Donner la taille de l'image.
	\item A quelle heure a été prise la photo?
	\item Relever la latitude et la longitude  du lieu de la prise.
\end{enumerate}

\bigskip

\textbf{Activité 2}

 En utilisant le programme donné dans l'activité 1 , récupérer les données GPS des deux images \textit{photo11.jpg et photo12.jpg} qui se trouvent dans le dossier \textit{Images}.\\

La page web \textit{\url{www.lexilogos.com/calcul\_distances.htm}} permet de calculer la distance qui sépare les deux endroits d'où ont été prises les deux photos.

\medskip


Déterminer cette distance.\\

\textbf{Activité3}

\medskip

L'instruction \textit{geodesic(ville1,ville2).miles} du module python  \textbf{geopy.distance} calcule la distance , en miles ,  entre deux endroits dont les coordonnées GPS (en décimal)sont données en arguments .

Pour déterminer les coordonnées GPS d'une ville vous pouvez utiliser le site \textit{\url{https://www.coordonnees-gps.fr/}} 
	
\begin{enumerate}
	\item 
	
	
	Ecrire une fonction "dist" qui demande à l'utilisateur les coordonnées de deux villes et qui retourne la distance entre ces deux villes.
	
    \begin{oldpy}
\begin{python}

# On importe la methode
from geopy.distance import geodesic

# Loading the lat-long data for Paris & Versailles
Paris = (48.862725, 2.287592)
Versailles = (48.8035403, 2.1266886)

# Print the distance calculated in km
print(geodesic(Paris, Versailles).km)	

	\end{python}
\end{oldpy}

\medskip

	\item Calculer les distances , en km , entre les villes d'\textit{Aigurande} et de  \textit{Dun le Palestel}  situés dans la creuse.
	
\end{enumerate}

\section{Notion d'image numérique}

\begin{tabular}{cc}
	\includegraphics[width=8cm,height=5cm]{Images/image014.jpg} 
	
	&
	
	\includegraphics[width=8cm,height=5cm]{Images/image005.png} \\
	
	une image de la nature   &  et sa représentation numérique
	
\end{tabular}


\vspace{1cm}


Visualisez d'abord  la vidéo suivante sur youtube :\\


\begin{center}
	\includegraphics[width=10cm,height=6cm]{Images/definition003.png}
	\centering{\url{https://www.youtube.com/watch?v=1iiRU76338U}}
\end{center}

\subsection{Définition et propriétés}
L'expression « image numérique » désigne toute image (dessin, icône, photographie…) acquise, créée, traitée et stockée sous \textbf{forme binaire}. \\


\begin{defi} Quelques définitions liées à l'image numérique

	\begin{itemize}
		\item 
		
		 \textbf{Définition} de l'image numérique = Nombre de pixels dans l'image.
		 
		 \vspace{0.3cm}
		
		\item \textbf{Taille} de l'image (matricielle) = Nombre de pixels par ligne et par colonne.
		
		\vspace{0.3cm}
		\item \textbf{Poids ou Volume } de l'image = Nombre d'octets nécessaires pour stocker l'image.
		
		\vspace{0.3cm}
		
		\item \textbf{Résolution} de l'image =  Nombre de pixels par unité de longueur (exprimée en \textbf{ppp}-pixel par pouce- pour une image numérique et en \textbf{dpi} -dot per inch ou point par pouce- pour l' imprimante).\\
		\textbf{1 pouce = 2,54 cm}
		
		\vspace{0.3cm}
		\item \textbf{Profondeur de couleur} = Nombre de bits nécessaires pour coder la couleur d'un pixel.
		
	\end{itemize}
\end{defi}	

\subsection {Différentes façons de coder une image }

\vspace{0.3cm}

\begin{description}
	\item[\RIGHTarrow] \textbf{Noir et blanc}: Chaque pixel est codé par 0 ou 1 (noir ou blanc).
	
	\vspace{0.3cm}
	\item[\RIGHTarrow] \textbf{Niveaux de gris}: Chaque pixel est codé par un entier compris entre 0 et 255 ( 255 nuances de gris par pixel).\\
	Combien de bits faut-il alors pour coder un pixel?
	
	\vspace{0.3cm}
	\item[\RIGHTarrow] \textbf{codage RVB}: Chaque pixel est codé par un triplet $(r,v,b)$ où $r,v,$ et $b$ sont des entiers compris entre 0 et 255.\\
	Chaque pixel est donc codé par 24 bits (combien en octets?)
\end{description}

\subsection{Différents formats du fichier image}
\arrayrulecolor{white}
\rowcolors[\hline]{1}{gray!25}{gray!25}
\begin{tabular}{|c|c|p{3cm}|m{2.5cm}|c|c|}
	\hline
	\rowcolor{lightgray}Format&Type& Compression&Nombre de couleurs &Animation & Transparence\\
	\hline
	JPEG&matriciel &  avec ou sans perte & 16 millions&oui &non\\
	\hline
	JPEG2000&matriciel &avec perte &32 millions &oui &oui\\
	\hline
	TIFF&matriciel &compression ou pas avec ou sans perte & de monochrome à 16 millions de couleurs& & \\
	\hline
	PNG&matriciel & & & & \\
	\hline
	SVG&Vectoriel & & & & \\
	\hline
\end{tabular}\\
 

\begin{tcolorbox}[width=17cm,enhanced,colback=lightgray!5!white,colframe=red,title=Pour approfondir,
	attach boxed title to top left=
	{yshift=-\tcboxedtitleheight/2},
	boxed title style={size=small,colback=red}]
	~~\\
	
	
	
	
	
	
	
	
	2 types d'images sont utilisés en informatique :\\
	Les images (matricielles) ou \textbf{bitmap}.\\
	
	
	\textbf{Une image matricielle} (ou bitmap) est une image constituée d'un ensemble de points :\\ les pixels.\\
	Chaque point porte des informations de position et de couleur.\\
	Format d'images bitmap : BMP, PCX, GIF, JPEG, TIFF.
	Les photos numériques et les images scannées sont de ce type.
	
	\medskip
	
	
	
	Les \textbf{images vectorielles}:
	
	\medskip
	
	Les images vectorielles sont composées de formes géométriques qui vont pouvoir être décrites d'un
	point de vue mathématique. Par exemple une droite sera définie par 2 points, un cercle par un
	centre et un rayon.\\ Le processeur est chargé de "traduire" ces formes en informations
	interprétables par la carte graphique (images Word, Publisher, CorelDraw - format WMF, CGM, etc.)\
	
	\medskip
	
	Les avantages d'une image vectorielle : les fichiers qui la composent sont petits, les re-
	dimensionnements sont faciles sans perte de qualité.\\
	Les inconvénients : une image vectorielle ne permet de représenter que des formes simples. Elle
	n'est pas donc utilisable pour la photographie notamment pour obtenir des photos réalistes.\\
	
	
	Les formats de fichiers des appareils photos numériques (APN)
	Les images sauvegardées sur la carte mémoire d'un APN sont toujours de type Bitmap. Le format
	le plus répandu est le format Jpeg. C'est un format de compression qui peut compresser les
	fichiers à plus de 90 .\\
	
	Les APN sont accompagnés de logiciels qui permettent de faire un choix du taux de compression.
	Plus le taux de compression est élevé plus l'image est altérée. Inversement un taux de
	compression moins grand donne une qualité d'image supérieure mais un fichier plus gros.
	
\end{tcolorbox}

\subsection{Activité autour de la résolution d'une image }


\begin{enumerate}

	\item Un photographe amateur fait appel à un prestataire sur net.
	
	\medskip
	
	Ce dernier envoient les photos avec une définition réduite à $1024 \times 786$.\\
	Le particulier souhaite les imprimer au format $12cm \times 9cm$.
	
	\medskip
	Quelle est la résolution de l'image avec ce format d'impression?
	\medskip
	
	\item 
	A Quelle définition le prestataire devrait envoyer les images si le client souhaite les imprimer  au format $12cm \times 9cm$ avec une résolution de 300ppp?
	
\end{enumerate}



\subsection{Création et visualisation d'images numériques : L'éditeur GIMP}

\vspace{0.3cm}
\textbf{En guise d'introduction}

\vspace{0.2cm}



\begin{enumerate}
	\item Réaliser un dessin sur une feuille quadrillée (petits carreaux) (dessiner dans un rectangle $20 \times 15$ carreaux).
	\item Coder votre dessin avec  des 0 et des 1  .
	\item Transmettre le dessin codé à votre voisin et demander lui de reconstituer le dessin.
\end{enumerate}


\vspace{0.4cm}


\textbf{Activité 1}

\vspace{0.2cm}


\begin{enumerate}
	\item Ouvrir le fichier image \textit{photo11.jpg} avec l'éditeur d'images \textit{GIMP}.
	\item Dans le menu \textbf{Image} pour afficher les propriétés de l'image.
	
	\begin{enumerate}
		\item 	Relever la taille , la définition et la résolution de l'image.
		\item Donner le volume de l'image .
		
	\end{enumerate}
\end{enumerate}

\vspace{0.4cm}


\textbf{Activité 2}








\vspace{0.2cm}


\begin{enumerate}
	\item Dans le menu \textbf{fichier} choisir \textbf{Nouvelle image} puis créer une une image de dimensions $90\times 50$.
	
	\vspace{0.2cm}
	\item Dans le menu \textbf{outils} , sélectionner  \textbf{crayon}.\\
	Choisir  une taille de 1 pixel ( les options de l'outil sont dans \textbf{Fenêtres> Fenêtres ancrables} et faire un dessin (noir sur blanc).
	
	\vspace{0.2cm}
	\item Enregistrer le dessin  (\textit{Mondessin01.pbm}) .\\
	Choisir dans le menu \textbf{fichier} l'option \textbf{exporter}.\\
	Sélectionner le format \textbf{pbm} (à la question "formatage des données" choisir ASCII).
	
	\vspace{0.2cm}
	
	
	\item Ouvrir le fichier avec un éditeur de texte.\\
	Décrire les données du fichier.
\end{enumerate}

\subsection{Application d’un filtre permettant la détection de contours dans GIMP}

\vspace{0.3cm}

\textbf{Activité 3}

\vspace{0.2cm}

\begin{enumerate}
	\item 
	
	Charger l'image "papillon.png"  dans le logiciel.
	
	\medskip
	
	
	\item 
	Obtention d’une image en niveaux de gris:
	
	\medskip
	
	 \textbf{menu Image > Mode > Niveaux de gris} 
	 
	 \medskip
	 
	 ou \\
	 
	 \textbf{Menu Couleurs > Désaturer} pour conserver les trois composantes afin de pouvoir les retravailler séparément.
	 
	 \medskip
	
	\item  Détection des contours: \textbf{menu Filtres > Génériques > Matrice de convolution}
	
	

\begin{center}
	
	\includegraphics[width=5cm,height=7cm]{Images/contour1.png}
	{\textit{contour avec l'option "Normaliser"}}
	
\end{center}

\item Essayer sans l'option "Normaliser"

\end{enumerate}

\textbf{Activité 3}


\vspace{0.4cm}
%\begin{tcolorbox}[title=Création d'une image numérique (formats )]
	\medskip
	
	
	%\begin{tcolorbox}[title=image en noir et blanc,nobeforeafter,inherit height]
	\begin{enumerate}	
		
	\item \textbf{Image en noir et blanc}
	
	\medskip
	
	
		\begin{enumerate}
			
			
			\item Ouvrir le fichier "dessin-01.pgm" .
			
			
			
			Sur la première ligne , "P1" désigne un type de fichier image (noir et blanc).\\
			
			La ligne suivante commence par le symbole $\#$.C'est un commentaire.\\
			
			Sur la ligne suivante les deux nombres entiers séparés par un espace désignent respectivement le nombre de lignes et le nombre de colonnes de l'image .\\
			
						
			La suite des données (des zéros et des 1) décrivent l'image numérique proprement dite)
			
			\item Modifier les données pour que l'image ressemble le plus possible au dessin que vous avez réalisé sur papier.\\
			
			Enregistrer le fichier en lui donnant le nom "MonDessin01.pgm".
			
			\medskip
			
			
			\item Visualiser le fichier.
			
			
		\end{enumerate}
		
%	\end{tcolorbox}
	\medskip
	\item  \textbf{Image en niveaux de gris}:
	
 \medskip
 
		
		A partir du fichier image "MonDessin01.pbm", vous allez fabriquer une image numérique où chaque pixel est codé à l'aide d'un entier compris entre 0 et 255 (les nuances de gris).
		\begin{enumerate}
			\item Ouvrir le fichier et remplacer "P1" par "P2".\\
			
			Après la ligne contenant les données sur la définition de l'image , ajouter une ligne sur laquelle vous marquez 255.\\
			
			
			Cette information donne le maximum de nuances de gris par pixel.
			
			\medskip
			
			\item Modifier alors les données numériques pour obtenir la "même image" avec des nuances de gris.\\
			
			
			Enregistrer le fichier (utiliser "MonDessin02.pnm" par exemple)
		\end{enumerate}
	%\end{tcolorbox}
	
	
	\medskip
	
%	\begin{tcolorbox}[title=image en couleur,coltitle=orange,colback=green!5,inherit height]
		
	\item  \textbf{Image en couleur}
	
	\medskip
	
	
		\begin{enumerate}
			\item Dans une image en couleur , chaque pixel est codé à l'aide de trois entiers compris entre 0 et 255:\\
			
			
			Ils représentent les nuances (ou intensité) du rouge ,du vert et du bleu respectivement.\\
			
			
			Sur la première ligne , remplacer P2 par P3 dans le fichier précédent.
			
			\medskip
			
			\item Faire alors les modifications nécessaires pour obtenir une image en couleur .\\
			
			
			N'oubliez pas d'entregistrer les modifications sous "Mondessin03.ppm" .
			
		\end{enumerate}
	\end{enumerate}

\section{Traitement de l'image numérique en quelques exemples avec python}

\vspace{0.4cm}

Tous les programmes ci-dessous auront pour préambule les lignes suivantes:\\


\begin{oldpy}
	\begin{python}
	from PIL import Image  #On importe le module imag de la libraire PIL
	import math  #le mode \textit{math} pur des calculs mathématiques
	
	\end{python}
\end{oldpy}

\subsection{Transformer une image en noir et blanc}

\vspace{0.3cm}
\begin{oldpy}
	\begin{python}
def noir_blanc(image):
   im=Image.open(image)
   im.show()
   pix=im.load()
   n , p = im.size[0] , im.size[1]
   seuil = 127 #on définit un seuil

   for x in range(n):
      for y in range(p):

          if pix[x,y][0] > seuil or pix[x,y][0] > seuil or pix[x,y][0] > seuil :
              pix[x,y]=(255,255,255)
          else:
              pix[x,y]=(0,0,0)
   im.show()
   return
\end{python}
\end{oldpy}


\subsection{Transformer une image en niveaux de gris}

\vspace{0.3cm}

\begin{oldpy}
	\begin{python}
def niveau_gris():
   im=Image.open(image)
   im.show()
   pix=im.load()
   n , p = im.size[0] , im.size[1]
   for x in range(n):
      for y in range(p):
         m=int((pix[x,y][0]+pix[x,y][1]+pix[x,y][2])/3)

         pix[x,y]=(m,m,m)
   return
\end{python}
\end{oldpy}

\subsection{Appliquer un filtre}


	\vspace{0.3cm}

~~~~~~~~~~~~	\includegraphics[width=6cm,height=4cm]{images/Cpapillon.jpeg} 	\hspace{2cm}
		\includegraphics[width=6cm,height=4cm]{images/papillon-bleu.jpg}
	
		
%	image couleur & application d'un filtre bleu
	
%\end{tabular}

\vspace{0.3cm}
\begin{oldpy}
	\begin{python}
def filtrer(image):

   im=Image.open(image)
   im.show()
   pix=im.load()
   n , p = im.size[0] , im.size[1]

   for x in range(n):
      for y in range(p):
         pix[x,y]=(0,0,pix[x,y][0])  #On ne garde que la composante bleue
   im.show()
   return

\end{python}

\end{oldpy}




\begin{enumerate}
	\item Appliquer la fonction "filtre" à l'image "papillon.jpeg" du dossier "fichiereleve".
	\item Modifier la fonction pour qu'elle applique un filtre rouge.
	\item Modifier la fonction pour qu'elle prenne comme paramètre supplémentaire une couleur donnée et qui affiche l'image filtrée par cette couleur.
\end{enumerate}






\vspace{0.4cm}

\subsection{Obtenir le négatif de l'image}

\vspace{0.3cm}

\begin{oldpy}
	\begin{python}
	def inverser_couleur():	
	    for x in range(n):
		   for y in range(p):
		       pix[x,y]=(255-pix[x,y][0],255-pix[x,y][1],255-pix[x,y][2])
	\end{python}
	
\end{oldpy}




\end{document}
